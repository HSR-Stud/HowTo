\section{Vorbereitung}

\subsection{Git / Github}
\subsubsection{Git installieren und einrichten}
Quelle: \url{https://help.github.com/articles/set-up-git}
\subsubsection{Github einrichten: HSR-Stud Style}
Quelle: \url{https://help.github.com/articles/working-with-ssh-key-passphrases}
\subsubsection{Github einrichten: Github Style}
Quelle: \url{https://help.github.com/articles/set-up-git}
\subsubsection{Github einrichten: Gangnam Style}
Quelle: \url{http://www.youtube.com/watch?v=9bZkp7q19f0}

\subsection{\LaTeX-Installation}
\subsubsection{Windows}
\subsubsection{Mac}
\subsubsection{Linux}


\subsection{Editoren}
\subsubsection{Eclipse mit TexLipse-Plugin}

\subsubsection{Sublime Text 2 mit LaTeXTools}

\subsubsection{TeXnicCenter}
TeXnicCenter wird in zwei Versionen angeboten, von uns jedoch nicht empfohlen. Version 1 bietet keine Unterstützung für UTF-8, was erhebliche Nachteile mit
Umlauten mit sich bringt. Vorallem Projekte, welche sowohl auf Windows als auch auf Unixsystem kompilierbar sein müssen, sind mit dem TeXnicCenter Version 1 fast unmöglich.
TeXnicCenter 2 befindet sich (immer) noch im Alpha Stadium und ist im Moment noch nicht wirklich zu gebrauchen. Falls Version 2 irgendwann mal Stable erreichen sollte kann es als
alternative zu Eclipse in betracht gezogen werden.


\subsection{Viewer}

\subsubsection{Sumatra PDF}

\subsubsection{GostViewer}

\subsubsection{Adobe Reader}
Da der Adobe Reader  PDF-Dateien beim Öffnen mit einem Schreibschutz versieht ist er äusserst unpraktisch. Nutzung wird nicht empfohlen.

\subsubsection{Adobe Writer}

\subsubsection{MuPDF}

MuPDF ist ein sehr kleiner, schneller, in (sehr sauberem) C programmierter PDF-Viewer. Momentan
besitzt er keinerlei Kontrollelemente, sondern wird vollständig via Tastatur gesteuert. Dies macht
ihn aber ideal zur PDF-Vorschau von \LaTeX-Dokumenten, ohne störende Werkzeugleisten.

Um das aktuelle Dokument nach Änderungen in MuPDF neu zu laden, muss man enweder im PDF-Fenster
\keystroke{R} drücken oder ein \texttt{SIGHUP} Signal an den entsprechenden Prozess senden
(\texttt{killall -HUP mupdf}). Ein Auto-Reload-Feature via Interrupts wird wegen der
Platformkompatibilität leider (noch) nicht angeboten.

MuPDF gibts für Linux, Windows, Android und iOS: \url{http://www.mupdf.com/}
